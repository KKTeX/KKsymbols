\documentclass[luatex,fontsize=10pt,paper=b5,twoside]{jlreq}%
\usepackage{KKsymbols}
\usepackage[dvipsnames, svgnames, x11names]{xcolor}
\usepackage{hyperref}
\usepackage{fp}
\usepackage{listings}
\usepackage{caption}
\usepackage{luacode}
\lstset{
    basicstyle=\ttfamily\small,
    keywordstyle=\color{blue},
    commentstyle=\color{gray},
    stringstyle=\color{red},
    breaklines=true,
    breakatwhitespace=false,  
    columns=flexible           
}

\usepackage{hyperref} 
\hypersetup{
  luatex, pdfencoding=auto, 
  colorlinks=true,
  linkcolor=black,     
  citecolor=black,     
  urlcolor=DeepSkyBlue3,      
  pdfborder={0 0 0}, 
}

\title{\texttt{KKsymbols} Package Documentation}
\author{Kosei Kawaguchi a.k.a. KKTeX}
\date{Version 2.0.0 (2025/12/23)}
\begin{document}

\begin{titlepage}
  \maketitle
\end{titlepage}
\newpage
\tableofcontents
\newpage

\section{Outline in Japanese}
このパッケージは、既存のotfフォントに頼ることなく、「任意のフォント、任意の引数で」丸数字などの特殊記号を再現する目的で作成されています。

luatexja-otfにおける\verb|\ajMaru|などは、その設計上いたしかなないデメリットとして、早見表がないと使い物にならないというものがありました。しかし、本パッケージでは、一から特殊文字を設計し直すというという取り組みを行なっているため、そのようなメリットが解消されています。

{%
\fboxsep=0pt\fboxrule=.1pt
\makebox[10\zw][s]{ああ\kakko{あ}ああ}


\directlua{
  for i = 1, 26 do
    local char = i
    tex.print("\\fbox{\\maru{" .. char .. "}}")
  end
}


\directlua{
  for i = string.byte("a"), string.byte("z") do
    local char = string.char(i)
    tex.print("\\fbox{\\maru*{" .. char .. "}}")
  end
}


\directlua{
  for i = string.byte("A"), string.byte("Z") do
    local char = string.char(i)
    tex.print("\\fbox{\\maru{" .. char .. "}}")
  end
}

\directlua{
  for i = 1, 26 do
    local char = i
    tex.print("\\fbox{\\jegg{" .. char .. "}}")
  end
}


\directlua{
  for i = string.byte("a"), string.byte("z") do
    local char = string.char(i)
    tex.print("\\fbox{\\jegg{" .. char .. "}}")
  end
}

\directlua{
  for i = string.byte("a"), string.byte("z") do
    local char = string.char(i)
    tex.print("\\fbox{\\jegg*{" .. char .. "}}")
  end
}


\directlua{
  for i = string.byte("a"), string.byte("z") do
    local char = string.char(i)
    tex.print("\\fbox{\\kuromaru*{" .. char .. "}}")
  end
}


\directlua{
  for i = string.byte("A"), string.byte("Z") do
    local char = string.char(i)
    tex.print("\\fbox{\\kuromaru{" .. char .. "}}")
  end
}

\directlua{
  for i = string.byte("a"), string.byte("z") do
    local char = string.char(i)
    tex.print("\\fbox{\\nmaru*{" .. char .. "}}")
  end
}


\directlua{
  for i = string.byte("A"), string.byte("Z") do
    local char = string.char(i)
    tex.print("\\fbox{\\nmaru{" .. char .. "}}")
  end
}


\directlua{
  for i = string.byte("a"), string.byte("z") do
    local char = string.char(i)
    tex.print("\\fbox{\\hishi*{" .. char .. "}}")
  end
}


\directlua{
  for i = string.byte("A"), string.byte("Z") do
    local char = string.char(i)
    tex.print("\\fbox{\\hishi{" .. char .. "}}")
  end
}


\directlua{
  for i = string.byte("a"), string.byte("z") do
    local char = string.char(i)
    tex.print("\\fbox{\\kurohishi*{" .. char .. "}}")
  end
}


\directlua{
  for i = string.byte("A"), string.byte("Z") do
    local char = string.char(i)
    tex.print("\\fbox{\\kurohishi{" .. char .. "}}")
  end
}


\directlua{
  for i = string.byte("a"), string.byte("z") do
    local char = string.char(i)
    tex.print("\\fbox{\\maruhishi*{" .. char .. "}}")
  end
}


\directlua{
  for i = string.byte("A"), string.byte("Z") do
    local char = string.char(i)
    tex.print("\\fbox{\\maruhishi{" .. char .. "}}")
  end
}


\directlua{
  for i = string.byte("a"), string.byte("z") do
    local char = string.char(i)
    tex.print("\\fbox{\\kuromaruhishi*{" .. char .. "}}")
  end
}


\directlua{
  for i = string.byte("A"), string.byte("Z") do
    local char = string.char(i)
    tex.print("\\fbox{\\kuromaruhishi{" .. char .. "}}")
  end
}


\directlua{
  for i = string.byte("a"), string.byte("z") do
    local char = string.char(i)
    tex.print("\\fbox{\\seihou*{" .. char .. "}}")
  end
}


\directlua{
  for i = string.byte("A"), string.byte("Z") do
    local char = string.char(i)
    tex.print("\\fbox{\\seihou{" .. char .. "}}")
  end
}


\directlua{
  for i = string.byte("a"), string.byte("z") do
    local char = string.char(i)
    tex.print("\\fbox{\\kuroseihou*{" .. char .. "}}")
  end
}


\directlua{
  for i = string.byte("A"), string.byte("Z") do
    local char = string.char(i)
    tex.print("\\fbox{\\kuroseihou{" .. char .. "}}")
  end
}


\directlua{
  for i = string.byte("a"), string.byte("z") do
    local char = string.char(i)
    tex.print("\\fbox{\\seimaru*{" .. char .. "}}")
  end
}


\directlua{
  for i = string.byte("A"), string.byte("Z") do
    local char = string.char(i)
    tex.print("\\fbox{\\seimaru{" .. char .. "}}")
  end
}


\directlua{
  for i = string.byte("a"), string.byte("z") do
    local char = string.char(i)
    tex.print("\\fbox{\\kuroseimaru*{" .. char .. "}}")
  end
}


\directlua{
  for i = string.byte("A"), string.byte("Z") do
    local char = string.char(i)
    tex.print("\\fbox{\\kuroseimaru{" .. char .. "}}")
  end
}


\directlua{
  for i = 1,26 do
    local char = i
    tex.print("\\fbox{\\kakko{" .. char .. "}}")
  end
}


\directlua{
  for i = string.byte("a"), string.byte("z") do
    local char = string.char(i)
    tex.print("\\fbox{\\kakko*{" .. char .. "}}")
  end
}


\directlua{
  for i = string.byte("A"), string.byte("Z") do
    local char = string.char(i)
    tex.print("\\fbox{\\kakko{" .. char .. "}}")
  end
}%
}


\section{Acknowledgements / Credit}
In developing this package, I made extensive use of the advice I received from Yusuke Terada.

\section{Installation}
Place \texttt{KKsymbols.sty} in a directory where LaTeX can find it, e.g., your local \texttt{texmf} tree or alongside your document.

Dependencies:
\begin{itemize}
    \item \texttt{LuaLaTeX-ja}
    \item \texttt{tikz}
    \item \texttt{clac}
\end{itemize}

Load the package:

\begin{verbatim}
\usepackage{KKsymbols}
\end{verbatim}

\section{Caution}
Since this package internally calls \verb|\ltjghostbeforejachar| and \verb|\ltjghostafterjachar|, it can be used only in a LuaLaTeX environment.

\section{Commands}
\subsection{The maru series}
This package provides \verb|\maru|, \verb|\kuromaru|, and \verb|\nmaru|. Each of them takes one mandatory argument and no optional arguments. You can pass strings of any length and in any font as arguments.

In most cases, \verb|\maru{argument}| will meet your demands. However, only when you take lowercase alphabet in these commands, you must use star-command just like \verb|\maru*{m}|.

They are used as follows.

\begin{table}[h]
\centering
\caption{maru series}
\begin{tabular}{|c|c|c|c|c|c|}
\hline
argument & \texttt{\textbackslash maru} & \texttt{\textbackslash kuromaru} & \texttt{\textbackslash nmaru} & \texttt{\textbackslash jegg} & \texttt{\textbackslash jegg*} \\
\hline
1     & \maru{1}     & \kuromaru{1}     & \nmaru{1} & \jegg{1} & \jegg*{1}  \\
97    & \maru{97}    & \kuromaru{97}    & \nmaru{97} & \jegg{97} & \jegg*{97} \\
だ    & \maru{だ}    & \kuromaru{だ}    & \nmaru{だ} & \jegg{だ} & \jegg*{だ} \\
ばばば & \maru{ばばば} & \kuromaru{ばばば} & \nmaru{ばばば} & \jegg{ばばば} & \jegg*{ばばば} \\
m & \maru*{m} & \kuromaru*{m} & \nmaru*{m} & \jegg{m} & \jegg*{m} \\
Qjg & \maru{Qjg} & \kuromaru{Qjg} & \nmaru{Qjg} & \jegg{Qjg} & \jegg*{Qjg} \\
\hline
\end{tabular}
\end{table}

They behave as if they were single kanji or hiragana characters:

\fboxsep=0pt\fboxrule=0.1pt
\begin{quotation}
  あいう\fbox{\maru{あ}}\fbox{あ}いう\maru{1}\maru{2}\maru{3}あいうえお
\end{quotation}

The spacing between \verb|\maru| and other characters is adjusted using \verb|\ltjghostbeforejachar| and \verb|\ltjghostafterjachar| so that it behaves like hiragana or kanji.

\textbf{Naturally, these commands also work correctly in vertical writing environments. But only when they include math-mode (in vertical mode), they don't work correctly.}

When changing the font size using commands such as \verb|\Large|, each command is scaled proportionally according to the font size change:

\begin{quotation}
  \tiny \maru{あああ}\kuromaru{2222}\nmaru{亀}
  \normalsize \maru{あああ}\kuromaru{2222}\nmaru{亀}
  \Huge \maru{あああ}\kuromaru{2222}\nmaru{亀}
\end{quotation}

You can also change the current font:
\begin{quotation}
  \LARGE\maru{あいう}\kuromaru{午後}\nmaru{悟}
  \gtfamily \maru{あいう}\kuromaru{午後}\nmaru{悟}
\end{quotation}

\section{The seihou series}
The commands introduced below are used in exactly the same way as the maru series.
In most cases, \verb|\seihou{argument}| will meet your demands. However, only when you take lowercase alphabet in these commands, you must use star-command just like \verb|\seihou*{m}|.

\bigskip{%
\centering
\captionof{table}{seihou series}
\begin{tabular}{|c|c|c|}
\hline
argument & \texttt{\textbackslash seihou} & \texttt{\textbackslash kuroseihou} \\
\hline
1     & \seihou{1}     & \kuroseihou{1}     \\
97    & \seihou{97}    & \kuroseihou{97}    \\
だ    & \seihou{だ}    & \kuroseihou{だ}    \\
ばばば & \seihou{ばばば} & \kuroseihou{ばばば} \\
m & \seihou*{m} & \kuroseihou*{m} \\
Qjg & \seihou{Qjg} & \kuroseihou{Qjg} \\
\hline
\end{tabular}
}\bigskip

{%
\centering
\captionof{table}{seimaru series}
\begin{tabular}{|c|c|c|}
\hline
argument & \texttt{\textbackslash seimaru} & \texttt{\textbackslash kuroseimaru} \\
\hline
1     & \seimaru{1}     & \kuroseimaru{1}     \\
97    & \seimaru{97}    & \kuroseimaru{97}    \\
だ    & \seimaru{だ}    & \kuroseimaru{だ}    \\
ばばば & \seimaru{ばばば} & \kuroseimaru{ばばば} \\
m & \seimaru*{m} & \kuroseimaru*{m} \\
Qjg & \seimaru{Qjg} & \kuroseimaru{Qjg} \\
\hline
\end{tabular}
}\bigskip

{%
\centering
\captionof{table}{hishi series}
\begin{tabular}{|c|c|c|c|c|}
\hline
argument & \texttt{\textbackslash hishi} & \texttt{\textbackslash kurohishi} & \texttt{\textbackslash maruhishi} & \texttt{\textbackslash kuromaruhishi} \\
\hline
1     & \hishi{1}     & \kurohishi{1}     & \maruhishi{1}     & \kuromaruhishi{1} \\
97    & \hishi{97}    & \kurohishi{97}    & \maruhishi{97}    & \kuromaruhishi{97} \\
だ    & \hishi{だ}    & \kurohishi{だ}    & \maruhishi{だ}    & \kuromaruhishi{だ} \\
ばばば & \hishi{ばばば} & \kurohishi{ばばば} & \maruhishi{ばばば} & \kuromaruhishi{ばばば} \\
m & \hishi*{m} & \kurohishi*{m} & \maruhishi*{m} & \kuromaruhishi*{m} \\
Qjg & \hishi{Qjg} & \kurohishi{Qjg} & \maruhishi{Qjg} & \kuromaruhishi{Qjg} \\
\hline
\end{tabular}
}\bigskip


\section{The kakko series}
The commands introduced below are used in exactly the same way as the maru series.

In most cases, \verb|\kakko{argument}| will meet your demands. However, only when you take lowercase alphabet in these commands, you must use star-command just like \verb|\kakko*{m}|.

\bigskip{%
\centering
\captionof{table}{kakko series\maru{1}}
\begin{tabular}{|c|c|c|c|c|c|}
\hline
argument & \texttt{\textbackslash kakko} & \texttt{\textbackslash sumikakko} & \texttt{\textbackslash kakukakko} & \texttt{\textbackslash kikakko} & \texttt{\textbackslash ykakko} \\
\hline
1     & \kakko{1}     & \sumikakko{1}     & \kakukakko{1}     & \kikakko{1}     & \ykakko{1} \\
97    & \kakko{97}    & \sumikakko{97}    & \kakukakko{97}    & \kikakko{97}    & \ykakko{97} \\
だ    & \kakko{だ}    & \sumikakko{だ}    & \kakukakko{だ}    & \kikakko{だ}    & \ykakko{だ} \\
ばばば & \kakko{ばばば} & \sumikakko{ばばば} & \kakukakko{ばばば} & \kikakko{ばばば} & \ykakko{ばばば} \\
m & \kakko*{m} & \sumikakko*{m} & \kakukakko*{m} & \kikakko*{m} & \ykakko*{m} \\
Qjg & \kakko{Qjg} & \sumikakko{Qjg} & \kakukakko{Qjg} & \kikakko{Qjg} & \ykakko{Qjg} \\
\hline
\end{tabular}
}

\bigskip{%
\centering
\captionof{table}{kakko series\maru{2}}
\begin{tabular}{|c|c|c|c|c|c|c|}
\hline
argument & \texttt{\textbackslash nykakko} & \texttt{\textbackslash namikakko} & \texttt{\textbackslash kagikakko} & \texttt{\textbackslash nkagikakko} & \texttt{\textbackslash period} & \texttt{\textbackslash ichimoji} \\
\hline
1     & \nykakko{1}     & \namikakko{1}     & \kagikakko{1}     & \nkagikakko{1}     & \period{1} & \ichimoji{1} \\
97    & \nykakko{97}    & \namikakko{97}    & \kagikakko{97}    & \nkagikakko{97}    & \period{97} & \ichimoji{97} \\
だ    & \nykakko{だ}    & \namikakko{だ}    & \kagikakko{だ}    & \nkagikakko{だ}    & \period{だ} & \ichimoji{だ} \\
ばばば & \nykakko{ばばば} & \namikakko{ばばば} & \kagikakko{ばばば} & \nkagikakko{ばばば} & \period{ばばば} & \ichimoji{ばばば} \\
m & \nykakko*{m} & \namikakko*{m} & \kagikakko*{m} & \nkagikakko*{m} & \period*{m} & \ichimoji*{m} \\
Qjg & \nykakko{Qjg} & \namikakko{Qjg} & \kagikakko{Qjg} & \nkagikakko{Qjg} & \period{Qjg} & \ichimoji{Qjg} \\
\hline
\end{tabular}
}\bigskip


\section{License}

Released under the \href{https://www.latex-project.org/lppl/}{LaTeX Project Public License (LPPL) 1.3c}.

\section{Version History}

\begin{itemize}
    \item \textbf{v1.0.0 (2025/10/03)} --- Initial public release.
    \item \textbf{v1.0.1 (2025/10/04)} --- Added \verb|\slowcare|, and adjusted \verb|\dccare|.
    \item \textbf{v1.0.2 (2025/10/05)} --- Fixed a problem related to dependency environments.
    \item \textbf{v1.0.3 (2025/10/12)} --- Fixed a problem related to maru series.
    \item \textbf{v1.0.4 (2025/10/18)} --- Added \verb|\ichimoji| command.
    \item \textbf{v1.1.0 (2025/10/28)} --- Unify all commands to zenkaku width.
    \item \textbf{v1.1.1 (2025/11/10)} --- Modified the scaling system of \verb|\ichimoji|.
    \item \textbf{v2.0.0 (2025/12/23)} --- Renewed every scaling system in order to make the outputs very close to oft characters.
\end{itemize}

\section{Source Code}

\begin{lstlisting}
  \NeedsTeXFormat{LaTeX2e}
  \ProvidesPackage{KKsymbols}[2025/12/23, Version 2.0.0]
  \RequirePackage{luatexja-adjust}
  \RequirePackage{expl3}
  \RequirePackage{calc}
  \RequirePackage{tikz}
  \usetikzlibrary{shapes}

  \DeclareRobustCommand{\maru}{%
    \@ifstar{\maru@new@no}{\maru@new}%
  }

  \DeclareRobustCommand{\kuromaru}{%
    \@ifstar{\kuromaru@new@no}{\kuromaru@new}%
  }

  \DeclareRobustCommand{\nmaru}{%
    \@ifstar{\nmaru@new@no}{\nmaru@new}%
  }

  \DeclareRobustCommand{\jegg}{%
    \@ifstar{\jegg@new@black}{\jegg@new}%
  }

  \DeclareRobustCommand{\seihou}{%
    \@ifstar{\seihou@new@no}{\seihou@new}%
  }

  \DeclareRobustCommand{\kuroseihou}{%
    \@ifstar{\kuroseihou@new@no}{\kuroseihou@new}%
  }

  \DeclareRobustCommand{\seimaru}{%
    \@ifstar{\seimaru@new@no}{\seimaru@new}%
  }

  \DeclareRobustCommand{\kuroseimaru}{%
    \@ifstar{\kuroseimaru@new@no}{\kuroseimaru@new}%
  }

  \DeclareRobustCommand{\hishi}{%
    \@ifstar{\hishi@new@no}{\hishi@new}%
  }

  \DeclareRobustCommand{\kurohishi}{%
    \@ifstar{\kurohishi@new@no}{\kurohishi@new}%
  }

  \DeclareRobustCommand{\maruhishi}{%
    \@ifstar{\kurohishi@new@no}{\kurohishi@new}%
  }

  \DeclareRobustCommand{\kuromaruhishi}{%
    \@ifstar{\kuromaruhishi@new@no}{\kuromaruhishi@new}%
  }

  \DeclareRobustCommand{\ichimoji}{%
    \@ifstar{\ichimoji@new@no}{\ichimoji@new}%
  }

  \DeclareRobustCommand{\kakko@y@internal}[1]{\kakko@byouga@internal{(}{)}{#1}}
  \DeclareRobustCommand{\kakko@y@internal@no}[1]{\kakko@byouga@internal@no{(}{)}{#1}}
  \DeclareRobustCommand{\kakko@t@internal}[1]{\kakko@byouga@internal@tate{(}{)}{#1}}
  \DeclareRobustCommand{\kakko@t@internal@no}[1]{\kakko@byouga@internal@tate@no{(}{)}{#1}}
  \DeclareRobustCommand{\kakko}{%
    \@ifstar{\kakko@star}{\kakko@nostar}% 
  }
  \newcommand{\kakko@star}[1]{%
    \ifnum\ltjgetparameter{direction}=3\relax
      \kakko@t@internal@no{#1}%
    \else
      \kakko@y@internal@no{#1}%
    \fi
  }
  \newcommand{\kakko@nostar}[1]{%
    \ifnum\ltjgetparameter{direction}=3\relax
      \kakko@t@internal{#1}%
    \else
      \kakko@y@internal{#1}%
    \fi
  }

  \DeclareRobustCommand{\sumikakko@y@internal}[1]{\kakko@byouga@internal{【}{】}{#1}}
  \DeclareRobustCommand{\sumikakko@y@internal@no}[1]{\kakko@byouga@internal@no{【}{】}{#1}}
  \DeclareRobustCommand{\sumikakko@t@internal}[1]{\kakko@byouga@internal@tate{【}{】}{#1}}
  \DeclareRobustCommand{\sumikakko@t@internal@no}[1]{\kakko@byouga@internal@tate@no{【}{】}{#1}}
  \DeclareRobustCommand{\sumikakko}{%
    \@ifstar{\sumikakko@star}{\sumikakko@nostar}% 
  }
  \newcommand{\sumikakko@star}[1]{%
    \ifnum\ltjgetparameter{direction}=3\relax
      \sumikakko@t@internal@no{#1}%
    \else
      \sumikakko@y@internal@no{#1}%
    \fi
  }
  \newcommand{\sumikakko@nostar}[1]{%
    \ifnum\ltjgetparameter{direction}=3\relax
      \sumikakko@t@internal{#1}%
    \else
      \sumikakko@y@internal{#1}%
    \fi
  }

  \DeclareRobustCommand{\kakukakko@y@internal}[1]{\kakko@byouga@internal{[}{]}{#1}}
  \DeclareRobustCommand{\kakukakko@y@internal@no}[1]{\kakko@byouga@internal@no{[}{]}{#1}}
  \DeclareRobustCommand{\kakukakko@t@internal}[1]{\kakko@byouga@internal@tate{[}{]}{#1}}
  \DeclareRobustCommand{\kakukakko@t@internal@no}[1]{\kakko@byouga@internal@tate@no{[}{]}{#1}}
  \DeclareRobustCommand{\kakukakko}{%
    \@ifstar{\kakukakko@star}{\kakukakko@nostar}% 
  }
  \newcommand{\kakukakko@star}[1]{%
    \ifnum\ltjgetparameter{direction}=3\relax
      \kakukakko@t@internal@no{#1}%
    \else
      \kakukakko@y@internal@no{#1}%
    \fi
  }
  \newcommand{\kakukakko@nostar}[1]{%
    \ifnum\ltjgetparameter{direction}=3\relax
      \kakukakko@t@internal{#1}%
    \else
      \kakukakko@y@internal{#1}%
    \fi
  }

  \DeclareRobustCommand{\kikakko@y@internal}[1]{\kakko@byouga@internal{〔}{〕}{#1}}
  \DeclareRobustCommand{\kikakko@y@internal@no}[1]{\kakko@byouga@internal@no{〔}{〕}{#1}}
  \DeclareRobustCommand{\kikakko@t@internal}[1]{\kakko@byouga@internal@tate{〔}{〕}{#1}}
  \DeclareRobustCommand{\kikakko@t@internal@no}[1]{\kakko@byouga@internal@tate@no{〔}{〕}{#1}}
  \DeclareRobustCommand{\kikakko}{%
    \@ifstar{\kikakko@star}{\kikakko@nostar}% 
  }
  \newcommand{\kikakko@star}[1]{%
    \ifnum\ltjgetparameter{direction}=3\relax
      \kikakko@t@internal@no{#1}%
    \else
      \kikakko@y@internal@no{#1}%
    \fi
  }
  \newcommand{\kikakko@nostar}[1]{%
    \ifnum\ltjgetparameter{direction}=3\relax
      \kikakko@t@internal{#1}%
    \else
      \kikakko@y@internal{#1}%
    \fi
  }

  \DeclareRobustCommand{\ykakko@y@internal}[1]{\kakko@byouga@internal{〈}{〉}{#1}}
  \DeclareRobustCommand{\ykakko@y@internal@no}[1]{\kakko@byouga@internal@no{〈}{〉}{#1}}
  \DeclareRobustCommand{\ykakko@t@internal}[1]{\kakko@byouga@internal@tate{〈}{〉}{#1}}
  \DeclareRobustCommand{\ykakko@t@internal@no}[1]{\kakko@byouga@internal@tate@no{〈}{〉}{#1}}
  \DeclareRobustCommand{\ykakko}{%
    \@ifstar{\ykakko@star}{\ykakko@nostar}% 
  }
  \newcommand{\ykakko@star}[1]{%
    \ifnum\ltjgetparameter{direction}=3\relax
      \ykakko@t@internal@no{#1}%
    \else
      \ykakko@y@internal@no{#1}%
    \fi
  }
  \newcommand{\ykakko@nostar}[1]{%
    \ifnum\ltjgetparameter{direction}=3\relax
      \ykakko@t@internal{#1}%
    \else
      \ykakko@y@internal{#1}%
    \fi
  }

  \DeclareRobustCommand{\nykakko@y@internal}[1]{\kakko@byouga@internal{《}{》}{#1}}
  \DeclareRobustCommand{\nykakko@y@internal@no}[1]{\kakko@byouga@internal@no{《}{》}{#1}}
  \DeclareRobustCommand{\nykakko@t@internal}[1]{\kakko@byouga@internal@tate{《}{》}{#1}}
  \DeclareRobustCommand{\nykakko@t@internal@no}[1]{\kakko@byouga@internal@tate@no{《}{》}{#1}}
  \DeclareRobustCommand{\nykakko}{%
    \@ifstar{\nykakko@star}{\nykakko@nostar}% 
  }
  \newcommand{\nykakko@star}[1]{%
    \ifnum\ltjgetparameter{direction}=3\relax
      \nykakko@t@internal@no{#1}%
    \else
      \nykakko@y@internal@no{#1}%
    \fi
  }
  \newcommand{\nykakko@nostar}[1]{%
    \ifnum\ltjgetparameter{direction}=3\relax
      \nykakko@t@internal{#1}%
    \else
      \nykakko@y@internal{#1}%
    \fi
  }

  \DeclareRobustCommand{\namikakko@y@internal}[1]{\kakko@byouga@internal{{}{}}{#1}}
  \DeclareRobustCommand{\namikakko@y@internal@no}[1]{\kakko@byouga@internal@no{{}{}}{#1}}
  \DeclareRobustCommand{\namikakko@t@internal}[1]{\kakko@byouga@internal@tate{{}{}}{#1}}
  \DeclareRobustCommand{\namikakko@t@internal@no}[1]{\kakko@byouga@internal@tate@no{{}{}}{#1}}
  \DeclareRobustCommand{\namikakko}{%
    \@ifstar{\namikakko@star}{\namikakko@nostar}% 
  }
  \newcommand{\namikakko@star}[1]{%
    \ifnum\ltjgetparameter{direction}=3\relax
      \namikakko@t@internal@no{#1}%
    \else
      \namikakko@y@internal@no{#1}%
    \fi
  }
  \newcommand{\namikakko@nostar}[1]{%
    \ifnum\ltjgetparameter{direction}=3\relax
      \namikakko@t@internal{#1}%
    \else
      \namikakko@y@internal{#1}%
    \fi
  }

  \DeclareRobustCommand{\kagikakko@y@internal}[1]{\kakko@byouga@internal{「}{」}{#1}}
  \DeclareRobustCommand{\kagikakko@y@internal@no}[1]{\kakko@byouga@internal@no{「}{」}{#1}}
  \DeclareRobustCommand{\kagikakko@t@internal}[1]{\kakko@byouga@internal@tate{「}{」}{#1}}
  \DeclareRobustCommand{\kagikakko@t@internal@no}[1]{\kakko@byouga@internal@tate@no{「}{」}{#1}}
  \DeclareRobustCommand{\kagikakko}{%
    \@ifstar{\kagikakko@star}{\kagikakko@nostar}% 
  }
  \newcommand{\kagikakko@star}[1]{%
    \ifnum\ltjgetparameter{direction}=3\relax
      \kagikakko@t@internal@no{#1}%
    \else
      \kagikakko@y@internal@no{#1}%
    \fi
  }
  \newcommand{\kagikakko@nostar}[1]{%
    \ifnum\ltjgetparameter{direction}=3\relax
      \kagikakko@t@internal{#1}%
    \else
      \kagikakko@y@internal{#1}%
    \fi
  }

  \DeclareRobustCommand{\nkagikakko@y@internal}[1]{\kakko@byouga@internal{『}{』}{#1}}
  \DeclareRobustCommand{\nkagikakko@y@internal@no}[1]{\kakko@byouga@internal@no{『}{』}{#1}}
  \DeclareRobustCommand{\nkagikakko@t@internal}[1]{\kakko@byouga@internal@tate{『}{』}{#1}}
  \DeclareRobustCommand{\nkagikakko@t@internal@no}[1]{\kakko@byouga@internal@tate@no{『}{』}{#1}}
  \DeclareRobustCommand{\nkagikakko}{%
    \@ifstar{\nkagikakko@star}{\nkagikakko@nostar}% 
  }
  \newcommand{\nkagikakko@star}[1]{%
    \ifnum\ltjgetparameter{direction}=3\relax
      \nkagikakko@t@internal@no{#1}%
    \else
      \nkagikakko@y@internal@no{#1}%
    \fi
  }
  \newcommand{\nkagikakko@nostar}[1]{%
    \ifnum\ltjgetparameter{direction}=3\relax
      \nkagikakko@t@internal{#1}%
    \else
      \nkagikakko@y@internal{#1}%
    \fi
  }


  \DeclareRobustCommand{\period@y@internal}[1]{\kakko@byouga@internal{\relax}{.}{#1}}
  \DeclareRobustCommand{\period@y@internal@no}[1]{\kakko@byouga@internal@no{\relax}{.}{#1}}
  \DeclareRobustCommand{\period@t@internal}[1]{\kakko@byouga@internal@tate{\relax}{.}{#1}}
  \DeclareRobustCommand{\period@t@internal@no}[1]{\kakko@byouga@internal@tate@no{\relax}{.}{#1}}
  \DeclareRobustCommand{\period}{%
    \@ifstar{\period@star}{\period@nostar}% 
  }
  \newcommand{\period@star}[1]{%
    \ifnum\ltjgetparameter{direction}=3\relax
      \period@t@internal@no{#1}%
    \else
      \period@y@internal@no{#1}%
    \fi
  }
  \newcommand{\period@nostar}[1]{%
    \ifnum\ltjgetparameter{direction}=3\relax
      \period@t@internal{#1}%
    \else
      \period@y@internal{#1}%
    \fi
  }


  % 全体で使用
  \NewDocumentCommand{\shukushou@adj@new}{ m m m }{%
    \ifdim #1 < #2%
      \resizebox*{\width}{#2}{#3}%
    \else%
      \resizebox*{#2}{#2}{#3}%
    \fi%
  }
  \NewDocumentCommand{\shukushou@adj@new@no}{ m m m }{%
    \ifdim #1 < #2%
      #3%
    \else%
      \resizebox*{!}{#2}{#3}%
    \fi%
  }
  \NewDocumentCommand{\RotYoko}{O{0}}{%
    \def\KKsymbolsRot@val@y{#1}%
  }
  \NewDocumentCommand{\RotTate}{O{90}}{%
    \def\KKsymbolsRot@val@t{#1}%
  }
  \def\KKsymbolsRot@val@t{90}
  \def\KKsymbolsRot@val@y{0}
  \NewDocumentCommand{\rot@or@not@kksym}{}{%
    \ifnum\ltjgetparameter{direction}=3
      \def\rot@kksym@tateyoko{\KKsymbolsRot@val@t}%
    \else
      \def\rot@kksym@tateyoko{\KKsymbolsRot@val@y}%
    \fi
  }
  %%%


  % \maru
  \NewDocumentCommand{\vertical@adj@maru@new}{}{%
    \ifnum\ltjgetparameter{direction}=3
      -.5\zw%
    \else
      -.12\zw%
    \fi
  }
  \newlength{\maru@boxwidth@new}
  \newlength{\maru@boxwidth@inner}
  \newlength{\maru@var@yokohaba}
  \newlength{\maru@var@yokohaba@cal}
  \newlength{\maru@var@tate}
  \newlength{\maru@var@fuka}
  \newlength{\maru@var@zenkou}

  \NewDocumentCommand{\shukushou@adj@new@no@maru}{ m m m m }{%
    \ifdim #1 < #2%
      \ifdim #3 > #2%
        \resizebox*{#3}{!}{#4}%
      \else%
        \scalebox{.9}{#4}%
      \fi%
    \else%
      \resizebox*{!}{#2}{#4}%
    \fi%
  }

  \DeclareRobustCommand{\maru@new}[1]{%
    \rot@or@not@kksym%
    \setlength{\maru@boxwidth@new}{\zw}%
    \settowidth{\maru@var@yokohaba}{#1}%
    \pgfmathsetlength{\maru@boxwidth@inner}{0.87 * \maru@boxwidth@new / sqrt(2)}%
    \ltjghostbeforejachar%
    \hbox to \maru@boxwidth@new{%
    \raisebox{\vertical@adj@maru@new}{%
    \vbox to \maru@boxwidth@new{%
    \vss%
    \hbox to \maru@boxwidth@new{%
      \hss%
      \tikz[baseline=(char.base)]{%
        \node[
          shape=circle,
          line width=.04\maru@boxwidth@new,  
          minimum size=.9\maru@boxwidth@new, 
          draw,
          inner sep=.02\maru@boxwidth@new,
          rotate=\rot@kksym@tateyoko
        ] (char){%
        \vbox to \maru@boxwidth@inner{
          \vss%
          \hbox to \maru@boxwidth@inner{%
          \hss%
            \shukushou@adj@new{\maru@var@yokohaba}{\maru@boxwidth@inner}{#1}%
          \hss%
          }%
          \vss%
        }%
        };
      }%
      \hss%
    }%
    \vss%
    }%
    }%
    }%
    \ltjghostafterjachar%
  }
  \DeclareRobustCommand{\maru@new@no}[1]{%
    \rot@or@not@kksym%
    \setlength{\maru@boxwidth@new}{\zw}%
    \settowidth{\maru@var@yokohaba}{#1}%
    \settoheight{\maru@var@tate}{#1}%
    \settodepth{\maru@var@fuka}{#1}%
    \pgfmathsetlength{\maru@var@zenkou}{\maru@var@tate + \maru@var@fuka}%
    \pgfmathsetlength{\maru@var@yokohaba@cal}{0.85 * \maru@var@yokohaba}%
    \pgfmathsetlength{\maru@boxwidth@inner}{0.87 * \maru@boxwidth@new / sqrt(2)}%
    \ltjghostbeforejachar%
    \hbox to \maru@boxwidth@new{%
    \raisebox{\vertical@adj@maru@new}{%
    \vbox to \maru@boxwidth@new{%
    \vss%
    \hbox to \maru@boxwidth@new{%
      \hss%
      \tikz[baseline=(char.base)]{%
        \node[
          shape=circle,
          line width=.04\maru@boxwidth@new,  
          minimum size=.9\maru@boxwidth@new, 
          draw,
          inner sep=.02\maru@boxwidth@new,
          rotate=\rot@kksym@tateyoko
        ] (char){%
        \vbox to \maru@boxwidth@inner{
          \vss%
          \hbox to \maru@boxwidth@inner{%
          \hss%
            \shukushou@adj@new@no@maru{\maru@var@zenkou}{\maru@boxwidth@inner}{\maru@var@yokohaba@cal}{#1}%
          \hss%
          }%
          \vss%
        }%
        };
      }%
      \hss%
    }%
    \vss%
    }%
    }%
    }%
    \ltjghostafterjachar%
  }


  \DeclareRobustCommand{\kuromaru@new}[1]{%
    \rot@or@not@kksym%
    \setlength{\maru@boxwidth@new}{\zw}%
    \settowidth{\maru@var@yokohaba}{#1}%
    \pgfmathsetlength{\maru@boxwidth@inner}{0.87 * \maru@boxwidth@new / sqrt(2)}%
    \ltjghostbeforejachar%
    \hbox to \maru@boxwidth@new{%
    \raisebox{\vertical@adj@maru@new}{%
    \vbox to \maru@boxwidth@new{%
    \vss%
    \hbox to \maru@boxwidth@new{%
      \hss%
      \tikz[baseline=(char.base)]{%
        \node[
          shape=circle,
          fill=black,%%%
          line width=.04\maru@boxwidth@new,  
          minimum size=.9\maru@boxwidth@new, 
          draw,
          inner sep=.02\maru@boxwidth@new,
          rotate=\rot@kksym@tateyoko
        ] (char){%
        \vbox to \maru@boxwidth@inner{
          \vss%
          \hbox to \maru@boxwidth@inner{%
          \hss%
            \shukushou@adj@new{\maru@var@yokohaba}{\maru@boxwidth@inner}{%
              \textcolor{white}{#1}}%
          \hss%
          }%
          \vss%
        }%
        };
      }%
      \hss%
    }%
    \vss%
    }%
    }%
    }%
    \ltjghostafterjachar%
  }
  \DeclareRobustCommand{\kuromaru@new@no}[1]{%
    \rot@or@not@kksym%
    \setlength{\maru@boxwidth@new}{\zw}%
    \settowidth{\maru@var@yokohaba}{#1}%
    \settoheight{\maru@var@tate}{#1}%
    \settodepth{\maru@var@fuka}{#1}%
    \pgfmathsetlength{\maru@var@zenkou}{\maru@var@tate + \maru@var@fuka}%
    \pgfmathsetlength{\maru@var@yokohaba@cal}{0.85 * \maru@var@yokohaba}%
    \pgfmathsetlength{\maru@boxwidth@inner}{0.87 * \maru@boxwidth@new / sqrt(2)}%
    \ltjghostbeforejachar%
    \hbox to \maru@boxwidth@new{%
    \raisebox{\vertical@adj@maru@new}{%
    \vbox to \maru@boxwidth@new{%
    \vss%
    \hbox to \maru@boxwidth@new{%
      \hss%
      \tikz[baseline=(char.base)]{%
        \node[
          shape=circle,
          fill=black,%%%
          line width=.04\maru@boxwidth@new,  
          minimum size=.9\maru@boxwidth@new, 
          draw,
          inner sep=.02\maru@boxwidth@new,
          rotate=\rot@kksym@tateyoko
        ] (char){%
        \vbox to \maru@boxwidth@inner{
          \vss%
          \hbox to \maru@boxwidth@inner{%
          \hss%
            \shukushou@adj@new@no@maru{\maru@var@zenkou}{\maru@boxwidth@inner}{\maru@var@yokohaba@cal}{%
              \textcolor{white}{#1}}%
          \hss%
          }%
          \vss%
        }%
        };
      }%
      \hss%
    }%
    \vss%
    }%
    }%
    }%
    \ltjghostafterjachar%
  }


  \DeclareRobustCommand{\nmaru@new}[1]{%
    \rot@or@not@kksym%
    \setlength{\maru@boxwidth@new}{\zw}%
    \settowidth{\maru@var@yokohaba}{#1}%
    \pgfmathsetlength{\maru@boxwidth@inner}{0.85 * \maru@boxwidth@new / sqrt(2)}%
    \ltjghostbeforejachar%
    \hbox to \maru@boxwidth@new{%
    \raisebox{\vertical@adj@maru@new}{%
    \vbox to \maru@boxwidth@new{%
    \vss%
    \hbox to \maru@boxwidth@new{%
      \hss%
      \tikz[baseline=(char.base)]{%
        \node[
          shape=circle,
          line width=.03\maru@boxwidth@new,  
          minimum size=.88\maru@boxwidth@new, 
          draw, double, double distance=.03\zw,%%%
          inner sep=.06\maru@boxwidth@new,
          rotate=\rot@kksym@tateyoko,
          scale=.87
        ] (char){%
        \vbox to \maru@boxwidth@inner{
          \vss%
          \hbox to \maru@boxwidth@inner{%
          \hss%
            \shukushou@adj@new{\maru@var@yokohaba}{\maru@boxwidth@inner}{#1}%
          \hss%
          }%
          \vss%
        }%
        };
      }%
      \hss%
    }%
    \vss%
    }%
    }%
    }%
    \ltjghostafterjachar%
  }
  \DeclareRobustCommand{\nmaru@new@no}[1]{%
    \rot@or@not@kksym%
    \setlength{\maru@boxwidth@new}{\zw}%
    \settowidth{\maru@var@yokohaba}{#1}%
    \settoheight{\maru@var@tate}{#1}%
    \settodepth{\maru@var@fuka}{#1}%
    \pgfmathsetlength{\maru@var@zenkou}{\maru@var@tate + \maru@var@fuka}%
    \pgfmathsetlength{\maru@var@yokohaba@cal}{0.85 * \maru@var@yokohaba}%
    \pgfmathsetlength{\maru@boxwidth@inner}{0.82 * \maru@boxwidth@new / sqrt(2)}%
    \ltjghostbeforejachar%
    \hbox to \maru@boxwidth@new{%
    \raisebox{\vertical@adj@maru@new}{%
    \vbox to \maru@boxwidth@new{%
    \vss%
    \hbox to \maru@boxwidth@new{%
      \hss%
      \tikz[baseline=(char.base)]{%
        \node[
          shape=circle,
          line width=.03\maru@boxwidth@new,  
          minimum size=.88\maru@boxwidth@new, 
          draw, double, double distance=.03\zw,%%%
          inner sep=.06\maru@boxwidth@new,
          rotate=\rot@kksym@tateyoko,
          scale=.9
        ] (char){%
        \vbox to \maru@boxwidth@inner{%
          \vss%
          \hbox to \maru@boxwidth@inner{%
          \hss%
            \shukushou@adj@new@no@maru{\maru@var@zenkou}{\maru@boxwidth@inner}{\maru@var@yokohaba@cal}{#1}%
          \hss%
          }%
          \vss%
        }%
        };
      }%
      \hss%
    }%
    \vss%
    }%
    }%
    }%
    \ltjghostafterjachar%
  }


  \newlength{\jegg@boxwidth@inner@cal}%%%
  \DeclareRobustCommand{\jegg@new}[1]{%
    \rot@or@not@kksym%
    \setlength{\maru@boxwidth@new}{\zw}%
    \settowidth{\maru@var@yokohaba}{#1}%
    \pgfmathsetlength{\maru@boxwidth@inner}{0.87 * \maru@boxwidth@new / sqrt(2)}%
    \pgfmathsetlength{\jegg@boxwidth@inner@cal}{0.77 * \maru@boxwidth@inner}%%%
    \ltjghostbeforejachar%
    \hbox to \maru@boxwidth@new{%
    \raisebox{\vertical@adj@maru@new}{%
    \vbox to \maru@boxwidth@new{%
    \vss%
    \hbox to \maru@boxwidth@new{%
      \hss%
      \tikz[baseline=(char.base)]{%
        \node[
          shape=ellipse,
          line width=.04\maru@boxwidth@new,  
          minimum height=.9\maru@boxwidth@new, 
          minimum width=.7\maru@boxwidth@new, 
          draw,
          inner sep=0\maru@boxwidth@new,
          rotate=\rot@kksym@tateyoko
        ] (char){%
        \vbox to \maru@boxwidth@inner{
          \vss%
          \hbox to \jegg@boxwidth@inner@cal{%
          \hss%
            \shukushou@adj@new{\maru@var@yokohaba}{\jegg@boxwidth@inner@cal}{#1}%
          \hss%
          }%
          \vss%
        }%
        };
      }%
      \hss%
    }%
    \vss%
    }%
    }%
    }%
    \ltjghostafterjachar%
  }
  \DeclareRobustCommand{\jegg@new@black}[1]{%
    \rot@or@not@kksym%
    \setlength{\maru@boxwidth@new}{\zw}%
    \settowidth{\maru@var@yokohaba}{#1}%
    \pgfmathsetlength{\maru@boxwidth@inner}{0.87 * \maru@boxwidth@new / sqrt(2)}%
    \pgfmathsetlength{\jegg@boxwidth@inner@cal}{0.77 * \maru@boxwidth@inner}%%%
    \ltjghostbeforejachar%
    \hbox to \maru@boxwidth@new{%
    \raisebox{\vertical@adj@maru@new}{%
    \vbox to \maru@boxwidth@new{%
    \vss%
    \hbox to \maru@boxwidth@new{%
      \hss%
      \tikz[baseline=(char.base)]{%
        \node[
          shape=ellipse,
          line width=.04\maru@boxwidth@new,  
          minimum height=.9\maru@boxwidth@new, 
          minimum width=.7\maru@boxwidth@new, 
          draw,
          fill=black!70,
          inner sep=0\maru@boxwidth@new,
          rotate=\rot@kksym@tateyoko
        ] (char){%
        \vbox to \maru@boxwidth@inner{
          \vss%
          \hbox to \jegg@boxwidth@inner@cal{%
          \hss%
            \shukushou@adj@new{\maru@var@yokohaba}{\jegg@boxwidth@inner@cal}{#1}%
          \hss%
          }%
          \vss%
        }%
        };
      }%
      \hss%
    }%
    \vss%
    }%
    }%
    }%
    \ltjghostafterjachar%
  }
  %%%


  % \seihou
  \NewDocumentCommand{\vertical@adj@seihou@new}{}{%
    \ifnum\ltjgetparameter{direction}=3
      -.5\zw%
    \else
      -.12\zw%
    \fi
  }
  \newlength{\seihou@boxwidth@new}
  \newlength{\seihou@boxwidth@inner}
  \newlength{\seihou@var@yokohaba@cal}
  \newlength{\seihou@var@yokohaba}
  \newlength{\seihou@var@tate}
  \newlength{\seihou@var@fuka}
  \newlength{\seihou@var@zenkou}

  \NewDocumentCommand{\shukushou@adj@new@no@seihou}{ m m m m }{%
    \ifdim #1 < #2%
      \ifdim #3 > #2%
        \resizebox*{#3}{!}{#4}%
      \else%
        \scalebox{.9}{#4}%
      \fi%
    \else%
      \resizebox*{!}{#2}{#4}%
    \fi%
  }

  \DeclareRobustCommand{\seihou@new}[1]{%
    \rot@or@not@kksym%
    \setlength{\seihou@boxwidth@new}{\zw}%
    \settowidth{\seihou@var@yokohaba}{#1}%
    \pgfmathsetlength{\seihou@boxwidth@inner}{0.68 * \seihou@boxwidth@new}%
    \ltjghostbeforejachar%
    \hbox to \seihou@boxwidth@new{%
    \raisebox{\vertical@adj@maru@new}{%
    \vbox to \seihou@boxwidth@new{%
    \vss%
    \hbox to \seihou@boxwidth@new{%
      \hss%
      \tikz[baseline=(char.base)]{%
        \node[
          line width=.04\seihou@boxwidth@new,  
          minimum size=.85\seihou@boxwidth@new, 
          draw,
          inner sep=.1\seihou@boxwidth@new,
          rotate=\rot@kksym@tateyoko
        ] (char){%
        \vbox to \seihou@boxwidth@inner{
          \vss%
          \hbox to \seihou@boxwidth@inner{%
          \hss%
            \shukushou@adj@new{\seihou@var@yokohaba}{\seihou@boxwidth@inner}{#1}%
          \hss%
          }%
          \vss%
        }%
        };
      }%
      \hss%
    }%
    \vss%
    }%
    }%
    }%
    \ltjghostafterjachar%
  }
  \DeclareRobustCommand{\seihou@new@no}[1]{%
    \rot@or@not@kksym%
    \setlength{\seihou@boxwidth@new}{\zw}%
    \settowidth{\seihou@var@yokohaba}{#1}%
    \settoheight{\seihou@var@tate}{#1}%
    \settodepth{\seihou@var@fuka}{#1}%
    \pgfmathsetlength{\seihou@var@zenkou}{\seihou@var@tate + \seihou@var@fuka}%
    \pgfmathsetlength{\seihou@var@yokohaba@cal}{0.85 * \seihou@var@yokohaba}%
    \pgfmathsetlength{\seihou@boxwidth@inner}{0.68 * \seihou@boxwidth@new}%
    \ltjghostbeforejachar%
    \hbox to \seihou@boxwidth@new{%
    \raisebox{\vertical@adj@maru@new}{%
    \vbox to \seihou@boxwidth@new{%
    \vss%
    \hbox to \seihou@boxwidth@new{%
      \hss%
      \tikz[baseline=(char.base)]{%
        \node[
          line width=.04\seihou@boxwidth@new,  
          minimum size=.85\seihou@boxwidth@new, 
          draw,
          inner sep=.1\seihou@boxwidth@new,
          rotate=\rot@kksym@tateyoko
        ] (char){%
        \vbox to \seihou@boxwidth@inner{
          \vss%
          \hbox to \seihou@boxwidth@inner{%
          \hss%
            \shukushou@adj@new@no@seihou{\seihou@var@zenkou}{\seihou@boxwidth@inner}{\seihou@var@yokohaba@cal}{#1}%
          \hss%
          }%
          \vss%
        }%
        };
      }%
      \hss%
    }%
    \vss%
    }%
    }%
    }%
    \ltjghostafterjachar%
  }


  \DeclareRobustCommand{\kuroseihou@new}[1]{%
    \rot@or@not@kksym%
    \setlength{\seihou@boxwidth@new}{\zw}%
    \settowidth{\seihou@var@yokohaba}{#1}%
    \pgfmathsetlength{\seihou@boxwidth@inner}{0.68 * \seihou@boxwidth@new}%
    \ltjghostbeforejachar%
    \hbox to \seihou@boxwidth@new{%
    \raisebox{\vertical@adj@maru@new}{%
    \vbox to \seihou@boxwidth@new{%
    \vss%
    \hbox to \seihou@boxwidth@new{%
      \hss%
      \tikz[baseline=(char.base)]{%
        \node[
          line width=.04\seihou@boxwidth@new,  
          minimum size=.85\seihou@boxwidth@new, 
          draw,
          fill=black,%%%
          inner sep=.1\seihou@boxwidth@new,
          rotate=\rot@kksym@tateyoko
        ] (char){%
        \vbox to \seihou@boxwidth@inner{
          \vss%
          \hbox to \seihou@boxwidth@inner{%
          \hss%
            \shukushou@adj@new{\seihou@var@yokohaba}{\seihou@boxwidth@inner}{%
              \textcolor{white}{#1}}%
          \hss%
          }%
          \vss%
        }%
        };
      }%
      \hss%
    }%
    \vss%
    }%
    }%
    }%
    \ltjghostafterjachar%
  }
  \DeclareRobustCommand{\kuroseihou@new@no}[1]{%
    \rot@or@not@kksym%
    \setlength{\seihou@boxwidth@new}{\zw}%
    \settowidth{\seihou@var@yokohaba}{#1}%
    \settoheight{\seihou@var@tate}{#1}%
    \settodepth{\seihou@var@fuka}{#1}%
    \pgfmathsetlength{\seihou@var@zenkou}{\seihou@var@tate + \seihou@var@fuka}%
    \pgfmathsetlength{\seihou@var@yokohaba@cal}{0.85 * \seihou@var@yokohaba}%
    \pgfmathsetlength{\seihou@boxwidth@inner}{0.68 * \seihou@boxwidth@new}%
    \ltjghostbeforejachar%
    \hbox to \seihou@boxwidth@new{%
    \raisebox{\vertical@adj@maru@new}{%
    \vbox to \seihou@boxwidth@new{%
    \vss%
    \hbox to \seihou@boxwidth@new{%
      \hss%
      \tikz[baseline=(char.base)]{%
        \node[
          line width=.04\seihou@boxwidth@new,  
          minimum size=.85\seihou@boxwidth@new, 
          draw,
          fill=black,%%%
          inner sep=.1\seihou@boxwidth@new,
          rotate=\rot@kksym@tateyoko
        ] (char){%
        \vbox to \seihou@boxwidth@inner{
          \vss%
          \hbox to \seihou@boxwidth@inner{%
          \hss%
            \shukushou@adj@new@no@seihou{\seihou@var@zenkou}{\seihou@boxwidth@inner}{\seihou@var@yokohaba@cal}{%
              \textcolor{white}{#1}}%
          \hss%
          }%
          \vss%
        }%
        };
      }%
      \hss%
    }%
    \vss%
    }%
    }%
    }%
    \ltjghostafterjachar%
  }


  \DeclareRobustCommand{\seimaru@new}[1]{%
    \rot@or@not@kksym%
    \setlength{\seihou@boxwidth@new}{\zw}%
    \settowidth{\seihou@var@yokohaba}{#1}%
    \pgfmathsetlength{\seihou@boxwidth@inner}{0.68 * \seihou@boxwidth@new}%
    \ltjghostbeforejachar%
    \hbox to \seihou@boxwidth@new{%
    \raisebox{\vertical@adj@maru@new}{%
    \vbox to \seihou@boxwidth@new{%
    \vss%
    \hbox to \seihou@boxwidth@new{%
      \hss%
      \tikz[baseline=(char.base)]{%
        \node[
          line width=.04\seihou@boxwidth@new,  
          minimum size=.85\seihou@boxwidth@new, 
          draw,
          rounded corners=.13\zw,%%%
          inner sep=.1\seihou@boxwidth@new,
          rotate=\rot@kksym@tateyoko
        ] (char){%
        \vbox to \seihou@boxwidth@inner{
          \vss%
          \hbox to \seihou@boxwidth@inner{%
          \hss%
            \shukushou@adj@new{\seihou@var@yokohaba}{\seihou@boxwidth@inner}{#1}%
          \hss%
          }%
          \vss%
        }%
        };
      }%
      \hss%
    }%
    \vss%
    }%
    }%
    }%
    \ltjghostafterjachar%
  }
  \DeclareRobustCommand{\seimaru@new@no}[1]{%
    \rot@or@not@kksym%
    \setlength{\seihou@boxwidth@new}{\zw}%
    \settowidth{\seihou@var@yokohaba}{#1}%
    \settoheight{\seihou@var@tate}{#1}%
    \settodepth{\seihou@var@fuka}{#1}%
    \pgfmathsetlength{\seihou@var@zenkou}{\seihou@var@tate + \seihou@var@fuka}%
    \pgfmathsetlength{\seihou@var@yokohaba@cal}{0.85 * \seihou@var@yokohaba}%
    \pgfmathsetlength{\seihou@boxwidth@inner}{0.68 * \seihou@boxwidth@new}%
    \ltjghostbeforejachar%
    \hbox to \seihou@boxwidth@new{%
    \raisebox{\vertical@adj@maru@new}{%
    \vbox to \seihou@boxwidth@new{%
    \vss%
    \hbox to \seihou@boxwidth@new{%
      \hss%
      \tikz[baseline=(char.base)]{%
        \node[
          line width=.04\seihou@boxwidth@new,  
          minimum size=.85\seihou@boxwidth@new, 
          draw,
          rounded corners=.13\zw,%%%
          inner sep=.1\seihou@boxwidth@new,
          rotate=\rot@kksym@tateyoko
        ] (char){%
        \vbox to \seihou@boxwidth@inner{
          \vss%
          \hbox to \seihou@boxwidth@inner{%
          \hss%
            \shukushou@adj@new@no@seihou{\seihou@var@zenkou}{\seihou@boxwidth@inner}{\seihou@var@yokohaba@cal}{#1}%
          \hss%
          }%
          \vss%
        }%
        };
      }%
      \hss%
    }%
    \vss%
    }%
    }%
    }%
    \ltjghostafterjachar%
  }


  \DeclareRobustCommand{\kuroseimaru@new}[1]{%
    \rot@or@not@kksym%
    \setlength{\seihou@boxwidth@new}{\zw}%
    \settowidth{\seihou@var@yokohaba}{#1}%
    \pgfmathsetlength{\seihou@boxwidth@inner}{0.68 * \seihou@boxwidth@new}%
    \ltjghostbeforejachar%
    \hbox to \seihou@boxwidth@new{%
    \raisebox{\vertical@adj@maru@new}{%
    \vbox to \seihou@boxwidth@new{%
    \vss%
    \hbox to \seihou@boxwidth@new{%
      \hss%
      \tikz[baseline=(char.base)]{%
        \node[
          line width=.04\seihou@boxwidth@new,  
          minimum size=.85\seihou@boxwidth@new, 
          draw,
          fill=black,%%%
          rounded corners=.13\zw,%%%
          inner sep=.1\seihou@boxwidth@new,
          rotate=\rot@kksym@tateyoko
        ] (char){%
        \vbox to \seihou@boxwidth@inner{
          \vss%
          \hbox to \seihou@boxwidth@inner{%
          \hss%
            \shukushou@adj@new{\seihou@var@yokohaba}{\seihou@boxwidth@inner}{%
              \textcolor{white}{#1}}%
          \hss%
          }%
          \vss%
        }%
        };
      }%
      \hss%
    }%
    \vss%
    }%
    }%
    }%
    \ltjghostafterjachar%
  }
  \DeclareRobustCommand{\kuroseimaru@new@no}[1]{%
    \rot@or@not@kksym%
    \setlength{\seihou@boxwidth@new}{\zw}%
    \settowidth{\seihou@var@yokohaba}{#1}%
    \settoheight{\seihou@var@tate}{#1}%
    \settodepth{\seihou@var@fuka}{#1}%
    \pgfmathsetlength{\seihou@var@zenkou}{\seihou@var@tate + \seihou@var@fuka}%
    \pgfmathsetlength{\seihou@var@yokohaba@cal}{0.85 * \seihou@var@yokohaba}%
    \pgfmathsetlength{\seihou@boxwidth@inner}{0.68 * \seihou@boxwidth@new}%
    \ltjghostbeforejachar%
    \hbox to \seihou@boxwidth@new{%
    \raisebox{\vertical@adj@maru@new}{%
    \vbox to \seihou@boxwidth@new{%
    \vss%
    \hbox to \seihou@boxwidth@new{%
      \hss%
      \tikz[baseline=(char.base)]{%
        \node[
          line width=.04\seihou@boxwidth@new,  
          minimum size=.85\seihou@boxwidth@new, 
          draw,
          fill=black,%%%
          rounded corners=.13\zw,%%%
          inner sep=.1\seihou@boxwidth@new,
          rotate=\rot@kksym@tateyoko
        ] (char){%
        \vbox to \seihou@boxwidth@inner{
          \vss%
          \hbox to \seihou@boxwidth@inner{%
          \hss%
            \shukushou@adj@new@no@seihou{\seihou@var@zenkou}{\seihou@boxwidth@inner}{\seihou@var@yokohaba@cal}{%
              \textcolor{white}{#1}}%
          \hss%
          }%
          \vss%
        }%
        };
      }%
      \hss%
    }%
    \vss%
    }%
    }%
    }%
    \ltjghostafterjachar%
  }


  \DeclareRobustCommand{\hishi@new}[1]{%
    \rot@or@not@kksym%
    \setlength{\seihou@boxwidth@new}{\zw}%
    \settowidth{\seihou@var@yokohaba}{#1}%
    \pgfmathsetlength{\seihou@boxwidth@inner}{0.85 * \seihou@boxwidth@new / 2}%
    \ltjghostbeforejachar%
    \hbox to \seihou@boxwidth@new{%
    \raisebox{\vertical@adj@maru@new}{%
    \vbox to \seihou@boxwidth@new{%
    \vss%
    \hbox to \seihou@boxwidth@new{%
      \hss%
      \tikz[baseline=(char.base)]{%
        \node[
          line width=.04\seihou@boxwidth@new,  
          minimum size=.5\seihou@boxwidth@new, 
          draw,
          shape=diamond,
          inner sep=.03\seihou@boxwidth@new,
          rotate=\rot@kksym@tateyoko
        ] (char){%
        \vbox to \seihou@boxwidth@inner{
          \vss%
          \hbox to \seihou@boxwidth@inner{%
          \hss%
            \shukushou@adj@new{\seihou@var@yokohaba}{\seihou@boxwidth@inner}{#1}%
          \hss%
          }%
          \vss%
        }%
        };
      }%
      \hss%
    }%
    \vss%
    }%
    }%
    }%
    \ltjghostafterjachar%
  }
  \DeclareRobustCommand{\hishi@new@no}[1]{%
    \rot@or@not@kksym%
    \setlength{\seihou@boxwidth@new}{\zw}%
    \settowidth{\seihou@var@yokohaba}{#1}%
    \settoheight{\seihou@var@tate}{#1}%
    \settodepth{\seihou@var@fuka}{#1}%
    \pgfmathsetlength{\seihou@var@zenkou}{(\seihou@var@tate + \seihou@var@fuka)*99}%
    \pgfmathsetlength{\seihou@boxwidth@inner}{0.85 * \seihou@boxwidth@new / 2}%
    \ltjghostbeforejachar%
    \hbox to \seihou@boxwidth@new{%
    \raisebox{\vertical@adj@maru@new}{%
    \vbox to \seihou@boxwidth@new{%
    \vss%
    \hbox to \seihou@boxwidth@new{%
      \hss%
      \tikz[baseline=(char.base)]{%
        \node[
          line width=.04\seihou@boxwidth@new,  
          minimum size=.5\seihou@boxwidth@new, 
          draw,
          shape=diamond,
          inner sep=.03\seihou@boxwidth@new,
          rotate=\rot@kksym@tateyoko
        ] (char){%
        \vbox to \seihou@boxwidth@inner{
          \vss%
          \hbox to \seihou@boxwidth@inner{%
          \hss%
            % これだけ他のやつと違うので注意!
            \shukushou@adj@new{\seihou@var@yokohaba}{\seihou@boxwidth@inner}{#1}%
          \hss%
          }%
          \vss%
        }%
        };
      }%
      \hss%
    }%
    \vss%
    }%
    }%
    }%
    \ltjghostafterjachar%
  }


  \DeclareRobustCommand{\maruhishi@new}[1]{%
    \rot@or@not@kksym%
    \setlength{\seihou@boxwidth@new}{\zw}%
    \settowidth{\seihou@var@yokohaba}{#1}%
    \pgfmathsetlength{\seihou@boxwidth@inner}{0.85 * \seihou@boxwidth@new / 2}%
    \ltjghostbeforejachar%
    \hbox to \seihou@boxwidth@new{%
    \raisebox{\vertical@adj@maru@new}{%
    \vbox to \seihou@boxwidth@new{%
    \vss%
    \hbox to \seihou@boxwidth@new{%
      \hss%
      \tikz[baseline=(char.base)]{%
        \node[
          line width=.04\seihou@boxwidth@new,  
          minimum size=.5\seihou@boxwidth@new, 
          draw,
          rounded corners=.06\zw,%%%
          shape=diamond,
          inner sep=.03\seihou@boxwidth@new,
          rotate=\rot@kksym@tateyoko
        ] (char){%
        \vbox to \seihou@boxwidth@inner{
          \vss%
          \hbox to \seihou@boxwidth@inner{%
          \hss%
            \shukushou@adj@new{\seihou@var@yokohaba}{\seihou@boxwidth@inner}{#1}%
          \hss%
          }%
          \vss%
        }%
        };
      }%
      \hss%
    }%
    \vss%
    }%
    }%
    }%
    \ltjghostafterjachar%
  }
  \DeclareRobustCommand{\maruhishi@new@no}[1]{%
    \rot@or@not@kksym%
    \setlength{\seihou@boxwidth@new}{\zw}%
    \settowidth{\seihou@var@yokohaba}{#1}%
    \settoheight{\seihou@var@tate}{#1}%
    \settodepth{\seihou@var@fuka}{#1}%
    \pgfmathsetlength{\seihou@var@zenkou}{(\seihou@var@tate + \seihou@var@fuka)*99}%
    \pgfmathsetlength{\seihou@boxwidth@inner}{0.85 * \seihou@boxwidth@new / 2}%
    \ltjghostbeforejachar%
    \hbox to \seihou@boxwidth@new{%
    \raisebox{\vertical@adj@maru@new}{%
    \vbox to \seihou@boxwidth@new{%
    \vss%
    \hbox to \seihou@boxwidth@new{%
      \hss%
      \tikz[baseline=(char.base)]{%
        \node[
          line width=.04\seihou@boxwidth@new,  
          minimum size=.5\seihou@boxwidth@new, 
          draw,
          rounded corners=.06\zw,%%%
          shape=diamond,
          inner sep=.03\seihou@boxwidth@new,
          rotate=\rot@kksym@tateyoko
        ] (char){%
        \vbox to \seihou@boxwidth@inner{
          \vss%
          \hbox to \seihou@boxwidth@inner{%
          \hss%
            % これだけ他のやつと違うので注意!
            \shukushou@adj@new{\seihou@var@yokohaba}{\seihou@boxwidth@inner}{#1}%
          \hss%
          }%
          \vss%
        }%
        };
      }%
      \hss%
    }%
    \vss%
    }%
    }%
    }%
    \ltjghostafterjachar%
  }


  \DeclareRobustCommand{\kurohishi@new}[1]{%
    \rot@or@not@kksym%
    \setlength{\seihou@boxwidth@new}{\zw}%
    \settowidth{\seihou@var@yokohaba}{#1}%
    \pgfmathsetlength{\seihou@boxwidth@inner}{0.85 * \seihou@boxwidth@new / 2}%
    \ltjghostbeforejachar%
    \hbox to \seihou@boxwidth@new{%
    \raisebox{\vertical@adj@maru@new}{%
    \vbox to \seihou@boxwidth@new{%
    \vss%
    \hbox to \seihou@boxwidth@new{%
      \hss%
      \tikz[baseline=(char.base)]{%
        \node[
          line width=.04\seihou@boxwidth@new,  
          minimum size=.5\seihou@boxwidth@new, 
          draw,
          fill=black,%%%
          shape=diamond,
          inner sep=.03\seihou@boxwidth@new,
          rotate=\rot@kksym@tateyoko
        ] (char){%
        \vbox to \seihou@boxwidth@inner{
          \vss%
          \hbox to \seihou@boxwidth@inner{%
          \hss%
            \shukushou@adj@new{\seihou@var@yokohaba}{\seihou@boxwidth@inner}{%
              \textcolor{white}{#1}}%
          \hss%
          }%
          \vss%
        }%
        };
      }%
      \hss%
    }%
    \vss%
    }%
    }%
    }%
    \ltjghostafterjachar%
  }
  \DeclareRobustCommand{\kurohishi@new@no}[1]{%
    \rot@or@not@kksym%
    \setlength{\seihou@boxwidth@new}{\zw}%
    \settowidth{\seihou@var@yokohaba}{#1}%
    \settoheight{\seihou@var@tate}{#1}%
    \settodepth{\seihou@var@fuka}{#1}%
    \pgfmathsetlength{\seihou@var@zenkou}{(\seihou@var@tate + \seihou@var@fuka)*99}%
    \pgfmathsetlength{\seihou@boxwidth@inner}{0.85 * \seihou@boxwidth@new / 2}%
    \ltjghostbeforejachar%
    \hbox to \seihou@boxwidth@new{%
    \raisebox{\vertical@adj@maru@new}{%
    \vbox to \seihou@boxwidth@new{%
    \vss%
    \hbox to \seihou@boxwidth@new{%
      \hss%
      \tikz[baseline=(char.base)]{%
        \node[
          line width=.04\seihou@boxwidth@new,  
          minimum size=.5\seihou@boxwidth@new, 
          draw,
          fill=black,%%%
          shape=diamond,
          inner sep=.03\seihou@boxwidth@new,
          rotate=\rot@kksym@tateyoko
        ] (char){%
        \vbox to \seihou@boxwidth@inner{
          \vss%
          \hbox to \seihou@boxwidth@inner{%
          \hss%
            % これだけ他のやつと違うので注意!
            \shukushou@adj@new{\seihou@var@yokohaba}{\seihou@boxwidth@inner}{%
              \textcolor{white}{#1}}%
          \hss%
          }%
          \vss%
        }%
        };
      }%
      \hss%
    }%
    \vss%
    }%
    }%
    }%
    \ltjghostafterjachar%
  }


  \DeclareRobustCommand{\kuromaruhishi@new}[1]{%
    \rot@or@not@kksym%
    \setlength{\seihou@boxwidth@new}{\zw}%
    \settowidth{\seihou@var@yokohaba}{#1}%
    \pgfmathsetlength{\seihou@boxwidth@inner}{0.85 * \seihou@boxwidth@new / 2}%
    \ltjghostbeforejachar%
    \hbox to \seihou@boxwidth@new{%
    \raisebox{\vertical@adj@maru@new}{%
    \vbox to \seihou@boxwidth@new{%
    \vss%
    \hbox to \seihou@boxwidth@new{%
      \hss%
      \tikz[baseline=(char.base)]{%
        \node[
          line width=.04\seihou@boxwidth@new,  
          minimum size=.5\seihou@boxwidth@new, 
          draw,
          fill=black,%%%
          rounded corners=.06\zw,%%%
          shape=diamond,
          inner sep=.03\seihou@boxwidth@new,
          rotate=\rot@kksym@tateyoko
        ] (char){%
        \vbox to \seihou@boxwidth@inner{%
          \vss%
          \hbox to \seihou@boxwidth@inner{%
          \hss%
            \shukushou@adj@new{\seihou@var@yokohaba}{\seihou@boxwidth@inner}{%
              \textcolor{white}{#1}}%
          \hss%
          }%
          \vss%
        }%
        };
      }%
      \hss%
    }%
    \vss%
    }%
    }%
    }%
    \ltjghostafterjachar%
  }
  \DeclareRobustCommand{\kuromaruhishi@new@no}[1]{%
    \rot@or@not@kksym%
    \setlength{\seihou@boxwidth@new}{\zw}%
    \settowidth{\seihou@var@yokohaba}{#1}%
    \settoheight{\seihou@var@tate}{#1}%
    \settodepth{\seihou@var@fuka}{#1}%
    \pgfmathsetlength{\seihou@var@zenkou}{(\seihou@var@tate + \seihou@var@fuka)*99}%
    \pgfmathsetlength{\seihou@boxwidth@inner}{0.85 * \seihou@boxwidth@new / 2}%
    \ltjghostbeforejachar%
    \hbox to \seihou@boxwidth@new{%
    \raisebox{\vertical@adj@maru@new}{%
    \vbox to \seihou@boxwidth@new{%
    \vss%
    \hbox to \seihou@boxwidth@new{%
      \hss%
      \tikz[baseline=(char.base)]{%
        \node[
          line width=.04\seihou@boxwidth@new,  
          minimum size=.5\seihou@boxwidth@new, 
          draw,
          fill=black,%%%
          rounded corners=.06\zw,%%%
          shape=diamond,
          inner sep=.03\seihou@boxwidth@new,
          rotate=\rot@kksym@tateyoko
        ] (char){%
        \vbox to \seihou@boxwidth@inner{
          \vss%
          \hbox to \seihou@boxwidth@inner{%
          \hss%
            % これだけ他のやつと違うので注意!
            \shukushou@adj@new{\seihou@var@yokohaba}{\seihou@boxwidth@inner}{%
              \textcolor{white}{#1}}%
          \hss%
          }%
          \vss%
        }%
        };
      }%
      \hss%
    }%
    \vss%
    }%
    }%
    }%
    \ltjghostafterjachar%
  }
  %%%


  % \kakko
  \newlength{\kakko@boxwidth@new}
  \newlength{\kakko@boxwidth@inner}
  \newlength{\kakko@var@yokohaba}
  \newlength{\kakko@var@yokohaba@cal}
  \newlength{\kakko@var@tate}
  \newlength{\kakko@var@fuka}
  \newlength{\kakko@var@zenkou}

  \NewDocumentCommand{\shukushou@adj@new@no@kakko}{ m m m m }{%
    \ifdim #1 < #2%
      \ifdim #3 > #2%
        \resizebox*{#3}{!}{#4}%
      \else%
        \scalebox{.9}{#4}%
      \fi%
    \else%
      \resizebox*{!}{#2}{#4}%
    \fi%
  }

  \DeclareRobustCommand{\kakko@byouga@internal}[3]{% 横の方
    \setlength{\kakko@boxwidth@new}{\zw}%
    \settowidth{\kakko@var@yokohaba}{#3}%
    \pgfmathsetlength{\kakko@boxwidth@inner}{0.8 * \kakko@boxwidth@new}%
    \ltjghostbeforejachar%
      \hbox to \zw{\hss%
        \scalebox{.63}[1]{#1}\hspace*{-.1\zw}%
          \vbox to \kakko@boxwidth@inner{
          \vss%
          \hbox to \kakko@boxwidth@inner{%
          \hss%
            \shukushou@adj@new{\kakko@var@yokohaba}{\kakko@boxwidth@inner}{#3}%
          \hss%
          }%
          \vss%
        }%
        \hspace*{-.1\zw}\scalebox{.63}[1]{#2}%
      \hss}%
    \ltjghostafterjachar%
  }
  \DeclareRobustCommand{\kakko@byouga@internal@no}[3]{% 横の方
    \setlength{\kakko@boxwidth@new}{\zw}%
    \settowidth{\kakko@var@yokohaba}{#3}%
    \settoheight{\kakko@var@tate}{#3}%
    \settodepth{\kakko@var@fuka}{#3}%
    \pgfmathsetlength{\kakko@var@zenkou}{\kakko@var@tate + \kakko@var@fuka}%
    \pgfmathsetlength{\kakko@var@yokohaba@cal}{0.9 * \kakko@var@yokohaba}%
    \pgfmathsetlength{\kakko@boxwidth@inner}{0.8 * \kakko@boxwidth@new}%
    \ltjghostbeforejachar%
      \hbox to \zw{\hss%
        \scalebox{.63}[1]{#1}\hspace*{-.1\zw}%
          \vbox to \kakko@boxwidth@inner{
          \vss%
          \hbox to \kakko@boxwidth@inner{%
          \hss%
            \shukushou@adj@new@no@kakko{\kakko@var@zenkou}{\kakko@boxwidth@inner}{\kakko@var@yokohaba@cal}{#3}%
          \hss%
          }%
          \vss%
        }%
        \hspace*{-.1\zw}\scalebox{.63}[1]{#2}%
      \hss}%
    \ltjghostafterjachar%
  }

  \DeclareRobustCommand{\kakko@byouga@internal@tate}[3]{% 縦の方
    \rot@or@not@kksym%
    \ifnum\ltjgetparameter{direction}=3\pgfmathsetmacro{\rot@kksym@tateyoko}{\rot@kksym@tateyoko - 90}\fi%
    \setlength{\kakko@boxwidth@new}{\zw}%
    \settowidth{\kakko@var@yokohaba}{#3}%
    \pgfmathsetlength{\kakko@boxwidth@inner}{0.8 * \kakko@boxwidth@new}%
    \ltjghostbeforejachar%
      \raisebox{-.5\zw}{%
      \rotatebox[origin=c]{90}{%
      \hbox to \zw{\hss%
      \vbox to \zw{\vss%
        \hbox to \zw{\hss%
        \scalebox{.63}[1]{#1}\hspace*{-.1\zw}%
          \vbox to \kakko@boxwidth@inner{%
          \vss%
          \hbox to \kakko@boxwidth@inner{%
          \hss%
            \raisebox{-1.2\zw}{%
              \rotatebox[origin=c]{\rot@kksym@tateyoko}{%
                \shukushou@adj@new{\kakko@var@yokohaba}{\kakko@boxwidth@inner}{#3}%
              }%
            }%
          \hss%
          }%
          \vss%
        }%
        \hspace*{-.1\zw}\scalebox{.63}[1]{#2}\vspace*{.3\zw}%
      \hss}%
      \vss}%
      \hss}%
      }%
      }%
    \ltjghostafterjachar%
  }
  \DeclareRobustCommand{\kakko@byouga@internal@tate@no}[3]{% 縦の方
    \rot@or@not@kksym%
    \ifnum\ltjgetparameter{direction}=3\pgfmathsetmacro{\rot@kksym@tateyoko}{\rot@kksym@tateyoko - 90}\fi%
    \setlength{\kakko@boxwidth@new}{\zw}%
    \settowidth{\kakko@var@yokohaba}{#3}%
    \settoheight{\kakko@var@tate}{#3}%
    \settodepth{\kakko@var@fuka}{#3}%
    \pgfmathsetlength{\kakko@var@zenkou}{\kakko@var@tate + \kakko@var@fuka}%
    \pgfmathsetlength{\kakko@var@yokohaba@cal}{0.9 * \kakko@var@yokohaba}%
    \pgfmathsetlength{\kakko@boxwidth@inner}{0.8 * \kakko@boxwidth@new}%
    \ltjghostbeforejachar%
      \raisebox{-.5\zw}{%
      \rotatebox[origin=c]{90}{%
      \hbox to \zw{\hss%
      \vbox to \zw{\vss%
        \hbox to \zw{\hss%
        \scalebox{.63}[1]{#1}\hspace*{-.1\zw}%
          \vbox to \kakko@boxwidth@inner{%
          \vss%
          \hbox to \kakko@boxwidth@inner{%
          \hss%
            \raisebox{-1.6\zw}{%
              \vbox to \kakko@boxwidth@inner{%
              \vss%
              \hbox to \kakko@boxwidth@inner{%
              \hss%
              \rotatebox[origin=c]{\rot@kksym@tateyoko}{%
                \shukushou@adj@new@no@kakko{\kakko@var@zenkou}{\kakko@boxwidth@inner}{\kakko@var@yokohaba@cal}{#3}%
              }\hss}\vss}%
            }%
          \hss%
          }%
          \vss%
        }%
        \hspace*{-.1\zw}\scalebox{.63}[1]{#2}\vspace*{.3\zw}%
      \hss}%
      \vss}%
      \hss}%
      }%
      }%
    \ltjghostafterjachar%
  }
  %%%


  % \ichimoji
  \newlength{\ichimoji@yk@new}
  \newlength{\ichimoji@ht@new}
  \newlength{\ichimoji@dp@new}
  \newlength{\ichimoji@zenkou@new}
  \DeclareRobustCommand{\ichimoji@scaling}[1]{%
    \ifdim\ichimoji@yk@new > \zw%
      \pgfmathsetmacro{\ratio@ichimoji@yk}{\strip@pt\zw/\strip@pt\ichimoji@yk@new}%
      \scalebox{\ratio@ichimoji@yk}[1]{#1}%
    \else%
      #1%
    \fi%
  }
  \DeclareRobustCommand{\ichimoji@new}[1]{%
    \ifnum\ltjgetparameter{direction}=3%
      \ltjghostbeforejachar%
      \raisebox{-.5\zw}{\hbox to \zw {\hss%
      \vbox to \zw {\vss%
        \hbox to \zw {\hss%
          \resizebox*{\zw}{\zw}{#1}%
        \hss}%
      \vss}\hss}}%
      \ltjghostafterjachar%
    \else%
      \ltjghostbeforejachar%
      \raisebox{-.12\zw}{\hbox to \zw {\hss%
      \vbox to \zw {\vss%
        \hbox to \zw {\hss%
          \resizebox*{\zw}{\zw}{#1}%
        \hss}%
      \vss}\hss}}%
      \ltjghostafterjachar%
    \fi%
  }
  \DeclareRobustCommand{\ichimoji@new@no}[1]{%
    \settowidth{\ichimoji@yk@new}{#1}%
    \settoheight{\ichimoji@ht@new}{#1}%
    \settodepth{\ichimoji@dp@new}{#1}%
    \pgfmathsetlength{\ichimoji@zenkou@new}{\ichimoji@ht@new + \ichimoji@dp@new}%
    \ifnum\ltjgetparameter{direction}=3%
      \ltjghostbeforejachar%
      \raisebox{-.5\zw}{\hbox to \zw {\hss%
      \vbox to \zw {\vss%
        \hbox to \zw {\hss%
          \ichimoji@scaling{#1}%
        \hss}%
      \vss}\hss}}%
      \ltjghostafterjachar%
    \else%
      \ltjghostbeforejachar%
      \raisebox{-.12\zw}{\hbox to \zw {\hss%
      \vbox to \zw {\vss%
        \hbox to \zw {\hss%
          \ichimoji@scaling{#1}%
        \hss}%
      \vss}\hss}}%
      \ltjghostafterjachar%
    \fi%
  }
  %%%
  \endinput
\end{lstlisting}
\end{document}